\documentclass[aspectratio=169]{beamer}

% setup basic metainfo
\title{Balancing Space Complexity and Ambiguity in Superadditive Set Functions}

\input{template}
% \documentclass [14pt,xcolor=dvipsnames,aspectratio=169]{beamer} 
\usetheme{metropolis}
\setbeamertemplate{caption}{\raggedright\insertcaption\par}
\metroset{block=fill}

\definecolor{mDarkBrown}{RGB}{45, 8, 26}
\definecolor{mDarkTeal}{RGB}{45, 8, 26}
\definecolor{mLightBrown}{RGB}{229, 0, 111}
\definecolor{mLightGreen}{RGB}{229, 0, 111}

\author[1]{Filip \'{U}radn\'{i}k}
% \institute{Charles University, Prague, Czech Republic}
\date{September 6, 2024}

\input{english}

\usepackage[natbib=true,style=authoryear,backend=bibtex,useprefix=true,maxbibnames=4]{biblatex}
\addbibresource{bibliography.bib}
\setbeamertemplate{caption}{\raggedright\insertcaption\par}

\def\blue#1{{\color{blue} #1}}
\def\orange#1{{\color{orange} #1}}
\def\red#1{{\color{red} #1}}
\def\s{\mathcal{S}}

\begin{document}
\maketitle

\begin{frame}{Set Functions}
	\begin{itemize}
		\item<1-> Finite \emph{ground set} $ N = \left\{ 1, \ldots, n \right\} $.
		\item<2-> A \emph{(complete) set function} is then any function \[
			f: \pot N \to \R.
		\]
		\item<3-> Used in countless fields, including cooperative game theory, combinatorial auctions, matroids.
	\end{itemize}
\end{frame}

\begin{frame}{What Is Wrong?}
	\begin{itemize}[ ]
		\item<1-> A \emph{(complete) set function} is any function \[
				f: \pot N \to \R.
			\]
		\item<2-> The size of $ f $ is exponential in the size of the ground set.
		\item<3-> We limit ourselves to the set of \emph{known subsets} $ \k \subseteq \pot N $.
	\end{itemize}
	
\end{frame}

\begin{frame}{Some Necessary Constraints}
	We assume \emph{superadditivity} of $ f $: \[
		\left( \forall S,T \subseteq N, S \cap T = \emptyset \right)\qquad \fce{f}{S} + \fce{f}{T} \leq \fce{f}{S \cup T}.
	\]
	We further assume minimal information to be present: \[
		\k \supseteq \k_0,
	\]
	where $ \k_0 = \left\{ \emptyset, N \right\} \cup \left\{ \left\{ i \right\} \suchthat i \in N \right\} $.
\end{frame}

\begin{frame}{Superadditive Extensions -- Candidates for Real Values}
	    \vspace{1.5em}
		\begin{center}
		\begin{tikzpicture}[] %%[scale=4] ONLY changes distances, not the canvas
    % Define the coordinates of the vertices
    \coordinate (A) at (0, 0);
    \coordinate (B) at (4.3, 0);
    \coordinate (C) at (7, 3.5);
    \coordinate (D) at (5, 6);
    \coordinate (E) at (2, 5);
    \coordinate (F) at (0, 2.5);
    \coordinate (v) at (3, 3.5);
    \coordinate (vm) at (A);
    \coordinate (vM) at (7, 6);
    
    % Draw and fill the hexagon
		\filldraw[very thick, fill=white!85!mDarkBrown, draw=mDarkBrown] (A) -- (B) -- (C) node[anchor=north,yshift=-30]{$\mathbb{S}^n(f,\k)$}-- (D) -- (E) -- (F) -- cycle;

			    % \filldraw[mDarkBrown] (A) circle (1pt) node[anchor=north]{$A$};
			    % \filldraw[mDarkBrown] (B) circle (1pt) node[anchor=north]{$B$};
			    % \filldraw[mDarkBrown] (C) circle (1pt) node[anchor=west]{$C$};
			    % \filldraw[mDarkBrown] (D) circle (1pt) node[anchor=south]{$D$};
			    % \filldraw[mDarkBrown] (E) circle (1pt) node[anchor=south]{$E$};
			    % \filldraw[mDarkBrown] (F) circle (1pt) node[anchor=east]{$F$};

			    \filldraw[mDarkBrown] (v) circle (1.5pt) node[anchor=north,xshift=3]{${f}$};

			    \filldraw<2->[mDarkBrown] (vm) circle (1.5pt) node[anchor=north,xshift=-3]{${\underline f}$};
			    \filldraw<2->[mDarkBrown] (vM) circle (1.5pt) node[anchor=north,xshift=5,yshift=2]{${\bar f}$};
			    \draw<2->[densely dotted, very thick, draw=mDarkBrown] (7, 3) -- (vM) -- (3.5, 6);
			    \draw<3->[densely dotted, very thick, draw=mLightBrown] (vm) -- (vM);
		\end{tikzpicture}
	\end{center}
\end{frame}


\begin{frame}{Divergence}
	\begin{definition}[Divergence]
		Let $ f $ be a set function and $ \k \subseteq \k_0 $.
		Let $ \ell: \R^n \times \R^n \to \R^+_0 $.
		The \emph{divergence} is \[
			\fce{\Delta_\ell}{f, \k} \deq \fce{\ell}{\bar f_{\k}, \underline f_{\k}}.
		\]
	\end{definition}
	
	\vspace{2em}
	We only require $ \ell $ to satisfy the following: \[
		\left( \forall \k \supseteq \k_0 \right)\! \left( \forall S \subseteq N \right)\quad \fce{\Delta_\ell}{f, \k} \geq \fce{\Delta_\ell}{f, \k \cup S}.
	\]
\end{frame}

\begin{frame}{Reducing $ \Delta_\ell $ -- Setting}
	\begin{itemize}[ ]
		\item<2-> We have \alert<2>{$ f \sim \mathcal{F} $}, where $ \mathcal{F} $ is a distribution of superadditive set functions.
		\item<3-> We only \alert<3>{know $ \k \supseteq \k_0 $} values of it.
		\item<4-> We have a \alert<4>{budget $ \tau $} of how many values we can learn.
	\end{itemize}
\end{frame}

\begin{frame}{Reducing $ \Delta_\ell $ -- Offline Approach}
	In the simplest case, we can \alert{minimize the expected value}: \[
		\s^* = \argmin_{\s, \absolute{\s} = \tau} \E_{v \sim \mathcal{F}} \left[ \mathcal{G}(v,\k \cup \s) \right].
	\]

	\vspace{2em}
	We call this the \emph{Offline} approach.
\end{frame}

\begin{frame}{Reducing $ \Delta_\ell $ -- Online Approach}
	It is \textquote{inefficient} to learn all values at once.

	The \emph{Online} approach seeks to find a policy $ \pi $ which selects the next value to learn based on the values already known.

	A solution to the Online approach can be approximated using reinforcement learning (we use the PPO algorithm).
\end{frame}

\begin{frame}{Lower Bound on the Online Approach}
	The online approach tries to learn additional information about the function.

	The \emph{best} we can hope for is that it learns \emph{all} the information.

	We can thus lower bound the resulting $ \Delta_\ell $ by \[
		\fceb{\E_{f \sim \mathcal{F}}}{\min_{\s, \absolute{\s} = \tau} \Delta_\ell(f, \k \cup \s)}.
	\]
	We call this the \emph{Oracle} benchmark.
\end{frame}

\begin{frame}{Reducing $ \Delta_\ell $ -- Results for $ \mathcal{F} = \texttt{factory} (5) $}
	\begin{center}
		\includegraphics[width=.8\textwidth]{figures/l1_norm_predictible_factory5.pdf}
	\end{center}
\end{frame}

\begin{frame}{Supermodular Functions}
	\begin{definition}[Supermodular Function]
		A set function $ f $ is \emph{supermodular} $ \equiv $ \[
			\left( \forall S,T \subseteq N \right)\quad \fce{f}{S} + \fce{f}{T} \leq \fce{f}{S \cup T} + \fce{f}{S \cap T}.
		\]
	\end{definition}
	
\end{frame}

\begin{frame}{Reducing $ \Delta_\ell $ -- Results for $ \mathcal{F} = \texttt{supermodular} (5) $}
	\begin{center}
		\includegraphics[width=.8\textwidth]{figures/l1_norm_convex5.pdf}
	\end{center}
\end{frame}

\begin{frame}{Reducing $ \Delta_\ell $ -- \textsc{Largest Subsets} Heuristic}
	\begin{center}
		\includegraphics[width=.8\textwidth]{figures/l1_norm_convex_linear.pdf}
	\end{center}
\end{frame}

\section{Conclusion}

\begin{frame}[allowframebreaks]{References}
    \nocite{*}
    \printbibliography[heading=none]
\end{frame}

\section{Thank You!}

\end{document}
